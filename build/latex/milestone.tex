%% Generated by Sphinx.
\def\sphinxdocclass{report}
\documentclass[letterpaper,10pt,english]{sphinxmanual}
\ifdefined\pdfpxdimen
   \let\sphinxpxdimen\pdfpxdimen\else\newdimen\sphinxpxdimen
\fi \sphinxpxdimen=.75bp\relax
%% turn off hyperref patch of \index as sphinx.xdy xindy module takes care of
%% suitable \hyperpage mark-up, working around hyperref-xindy incompatibility
\PassOptionsToPackage{hyperindex=false}{hyperref}

\PassOptionsToPackage{warn}{textcomp}

\catcode`^^^^00a0\active\protected\def^^^^00a0{\leavevmode\nobreak\ }
\usepackage{cmap}
\usepackage{xeCJK}
\usepackage{amsmath,amssymb,amstext}
\usepackage{polyglossia}
\setmainlanguage{english}



\setmainfont{FreeSerif}[
  Extension      = .otf,
  UprightFont    = *,
  ItalicFont     = *Italic,
  BoldFont       = *Bold,
  BoldItalicFont = *BoldItalic
]
\setsansfont{FreeSans}[
  Extension      = .otf,
  UprightFont    = *,
  ItalicFont     = *Oblique,
  BoldFont       = *Bold,
  BoldItalicFont = *BoldOblique,
]
\setmonofont{FreeMono}[
  Extension      = .otf,
  UprightFont    = *,
  ItalicFont     = *Oblique,
  BoldFont       = *Bold,
  BoldItalicFont = *BoldOblique,
]


\usepackage[Sonny]{fncychap}
\ChNameVar{\Large\normalfont\sffamily}
\ChTitleVar{\Large\normalfont\sffamily}
\usepackage{sphinx}

\fvset{fontsize=\small}
\usepackage{geometry}

% Include hyperref last.
\usepackage{hyperref}
% Fix anchor placement for figures with captions.
\usepackage{hypcap}% it must be loaded after hyperref.
% Set up styles of URL: it should be placed after hyperref.
\urlstyle{same}

\usepackage{sphinxmessages}




\title{Milestone}
\date{2020 年 03 月 15 日}
\release{1.0}
\author{MA Yuhai}
\newcommand{\sphinxlogo}{\vbox{}}
\renewcommand{\releasename}{发布}
\makeindex
\begin{document}

\pagestyle{empty}
\sphinxmaketitle
\pagestyle{plain}
\sphinxtableofcontents
\pagestyle{normal}
\phantomsection\label{\detokenize{index::doc}}



\chapter{主要功能特性}
\label{\detokenize{intro/index:id1}}\label{\detokenize{intro/index::doc}}\begin{enumerate}
\sphinxsetlistlabels{\arabic}{enumi}{enumii}{}{.}%
\item {} 
用于模型在环仿真(MiL)的C/C++代码集成工具

\item {} 
简捷、统一,一次编码同时支持FMI及S-Function模型接口标准

\item {} 
使用CMake构建系统,可自动适配大多数编译器

\item {} 
使用Qt图形界面,可运行于大部分操作系统的桌面环境

\item {} 
与来自IBK的MasterSimulator环境集成发布,便于FMU集成仿真

\item {} 
生成平台相关的代码,兼容Windows及Linux

\item {} 
生成的模型可运行在基于Linux的半实物仿真(HiL)系统

\end{enumerate}


\chapter{安装与配置}
\label{\detokenize{install/index:id1}}\label{\detokenize{install/index::doc}}

\section{软件授权}
\label{\detokenize{install/_u8f6f_u4ef6_u5b89_u88c5:id1}}\label{\detokenize{install/_u8f6f_u4ef6_u5b89_u88c5::doc}}
提供运行系统的MAC地址,获取授权文件


\subsection{Windows}
\label{\detokenize{install/_u8f6f_u4ef6_u5b89_u88c5:windows}}
ipconfig /all


\subsection{Linux}
\label{\detokenize{install/_u8f6f_u4ef6_u5b89_u88c5:linux}}
ifconfig


\section{软件安装}
\label{\detokenize{install/_u8f6f_u4ef6_u5b89_u88c5:id2}}
工具包解压缩得到主目录结构
将授权文件放置在license目录中


\section{外部依赖环境配置}
\label{\detokenize{install/_u73af_u5883_u914d_u7f6e:id1}}\label{\detokenize{install/_u73af_u5883_u914d_u7f6e::doc}}

\subsection{编译器}
\label{\detokenize{install/_u73af_u5883_u914d_u7f6e:id2}}

\subsubsection{Windows}
\label{\detokenize{install/_u73af_u5883_u914d_u7f6e:windows}}
Visual Studio中的cl编译器,注意不要使用绿色安装。示例代码在VS2010及以上版本中经过测试,但推荐使用VS2013及以上的版本,以支持C99中的编码习惯。
Linux
\textasciicircum{}\textasciicircum{}\textasciicircum{}\textasciicircum{}\textasciicircum{}
gcc/g++或clang/clang++编译器,推荐在系统的包管理器中安装。


\subsection{CMake}
\label{\detokenize{install/_u73af_u5883_u914d_u7f6e:cmake}}
开发工具包执行需要部署cmake运行环境,env中包含相应的安装文件。
此外Windows下还要部署VC运行时环境,Linux下需要授予bin目录下程序执行权限(chmod 777 ./bin/{\color{red}\bfseries{}*})。


\subsection{路径配置}
\label{\detokenize{install/_u73af_u5883_u914d_u7f6e:id5}}
在MasterSim中选择当前系统中安装的CMake路径以及milestone可执行文件的路径。


\subsection{切换界面语言}
\label{\detokenize{install/_u73af_u5883_u914d_u7f6e:id6}}
在MasterSim中切换Milestone的界面语言,重新启动后生效。


\chapter{案例教程}
\label{\detokenize{tutorial/index:id1}}\label{\detokenize{tutorial/index::doc}}

\section{运行案例}
\label{\detokenize{tutorial/index:id2}}
在工具包根目录下执行”mkdir build” //建立单独的构建目录,名称任意,用于将临时文件与工具分开
在工具包根目录下执行”cd build” //切换到创建的构建目录
在创建的构建目录下执行”cmake ..” //在构建目录下,指定代码目录在上层目录,生成编译工程文件(Windows下为MSVC sln,Linux下为Makefile)
在创建的构建目录下执行”cmake \textendash{}build . ” //执行构建,注意”.”为当前目录,附加”\textendash{}config Release”或”\textendash{}config Debug”参数切换Release版和Debug版,默认为Debug版
所导出的fmu模型在export目录中
依次在不同系统下执行工具包,将model目录(保留已生成的中间文件)或整个工具包复制到其他系统继续构建,将获得同时支持多系统的fmu文件


\section{代码结构剖析}
\label{\detokenize{tutorial/index:id3}}
\begin{sphinxVerbatim}[commandchars=\\\{\},numbers=left,firstnumber=1,stepnumber=1]
\PYG{c+cp}{\PYGZsh{}}\PYG{c+cp}{ifndef INTERFACE\PYGZus{}H\PYGZus{}\PYGZus{}}
\PYG{c+cp}{\PYGZsh{}}\PYG{c+cp}{define INTERFACE\PYGZus{}H\PYGZus{}\PYGZus{}}
\PYG{c+c1}{//==================================================================/}
\PYG{c+c1}{// A test case for fmi simulation tools}
\PYG{c+c1}{// Copyright (c) 2019 马玉海}
\PYG{c+c1}{// All rights reserved.}
\PYG{c+c1}{//}
\PYG{c+c1}{// Version 1.0}
\PYG{c+c1}{//==================================================================/}
\PYG{c+cp}{\PYGZsh{}}\PYG{c+cp}{define \PYGZus{}CRT\PYGZus{}SECURE\PYGZus{}NO\PYGZus{}WARNINGS}
\PYG{c+cp}{\PYGZsh{}}\PYG{c+cp}{include} \PYG{c+cpf}{\PYGZlt{}math.h\PYGZgt{}}
\PYG{c+cp}{\PYGZsh{}}\PYG{c+cp}{include} \PYG{c+cpf}{\PYGZlt{}stdio.h\PYGZgt{}}
\PYG{c+cp}{\PYGZsh{}}\PYG{c+cp}{include} \PYG{c+cpf}{\PYGZlt{}stdlib.h\PYGZgt{}}
\PYG{c+cp}{\PYGZsh{}}\PYG{c+cp}{include} \PYG{c+cpf}{\PYGZlt{}memory.h\PYGZgt{}}
\PYG{c+cp}{\PYGZsh{}}\PYG{c+cp}{include} \PYG{c+cpf}{\PYGZlt{}string.h\PYGZgt{}}
\PYG{c+cp}{\PYGZsh{}}\PYG{c+cp}{include} \PYG{c+cpf}{\PYGZlt{}float.h\PYGZgt{}}

\PYG{c+cp}{\PYGZsh{}}\PYG{c+cp}{define IO\PYGZus{}PORT\PYGZus{}FLUSH(data\PYGZus{}type, var\PYGZus{}name) \PYGZbs{}}
\PYG{c+cp}{    do\PYGZob{}\PYGZbs{}}
\PYG{c+cp}{        memset(\PYGZam{}(p\PYGZhy{}\PYGZgt{}var\PYGZus{}name), 0, sizeof(data\PYGZus{}type));\PYGZbs{}}
\PYG{c+cp}{    \PYGZcb{} while(0);}

\PYG{c+cp}{\PYGZsh{}}\PYG{c+cp}{ifndef \PYGZus{}\PYGZus{}cplusplus}
\PYG{c+cp}{\PYGZsh{}}\PYG{c+cp}{define FMI\PYGZus{}EXPORT}
\PYG{c+cp}{\PYGZsh{}}\PYG{c+cp}{define bool unsigned char}
\PYG{c+cp}{\PYGZsh{}}\PYG{c+cp}{define true 1}
\PYG{c+cp}{\PYGZsh{}}\PYG{c+cp}{define false 0}
\PYG{c+cp}{\PYGZsh{}}\PYG{c+cp}{else}
\PYG{c+cp}{\PYGZsh{}}\PYG{c+cp}{define FMI\PYGZus{}EXPORT extern \PYGZdq{}C\PYGZdq{}}
\PYG{c+cp}{\PYGZsh{}}\PYG{c+cp}{endif}

\PYG{c+cp}{\PYGZsh{}}\PYG{c+cp}{ifndef \PYGZus{}WIN32}
\PYG{c+cp}{\PYGZsh{}}\PYG{c+cp}{include} \PYG{c+cpf}{\PYGZlt{}limits.h\PYGZgt{}}
\PYG{c+cp}{\PYGZsh{}}\PYG{c+cp}{define \PYGZus{}MAX\PYGZus{}PATH PATH\PYGZus{}MAX}
\PYG{c+cp}{\PYGZsh{}}\PYG{c+cp}{define \PYGZus{}MAX\PYGZus{}FNAME NAME\PYGZus{}MAX}
\PYG{c+cp}{\PYGZsh{}}\PYG{c+cp}{define \PYGZus{}MAX\PYGZus{}EXT NAME\PYGZus{}MAX}
\PYG{c+cp}{\PYGZsh{}}\PYG{c+cp}{endif}
\PYG{c+c1}{// non\PYGZhy{}standard interface definition}
\PYG{c+cp}{\PYGZsh{}}\PYG{c+cp}{define FMI\PYGZus{}IN}
\PYG{c+cp}{\PYGZsh{}}\PYG{c+cp}{define FMI\PYGZus{}OUT}
\PYG{c+cp}{\PYGZsh{}}\PYG{c+cp}{define FMI\PYGZus{}PRM}
\PYG{k}{typedef} \PYG{k}{const} \PYG{k+kt}{char} \PYG{o}{*} \PYG{n}{fmi\PYGZus{}str\PYGZus{}ptr}\PYG{p}{;}

\PYG{n}{FMI\PYGZus{}EXPORT} \PYG{k+kt}{void} \PYG{o}{*}\PYG{n+nf}{fmi\PYGZus{}instantiate}\PYG{p}{(}\PYG{k+kt}{void}\PYG{p}{)}\PYG{p}{;}
\PYG{n}{FMI\PYGZus{}EXPORT} \PYG{k+kt}{int} \PYG{n+nf}{fmi\PYGZus{}initialize}\PYG{p}{(}\PYG{k+kt}{void} \PYG{o}{*}\PYG{p}{)}\PYG{p}{;}
\PYG{n}{FMI\PYGZus{}EXPORT} \PYG{k+kt}{int} \PYG{n+nf}{fmi\PYGZus{}doStep}\PYG{p}{(}\PYG{k+kt}{void} \PYG{o}{*}\PYG{p}{)}\PYG{p}{;}
\PYG{n}{FMI\PYGZus{}EXPORT} \PYG{k+kt}{int} \PYG{n+nf}{fmi\PYGZus{}reset}\PYG{p}{(}\PYG{k+kt}{void} \PYG{o}{*}\PYG{p}{)}\PYG{p}{;}
\PYG{n}{FMI\PYGZus{}EXPORT} \PYG{k+kt}{void} \PYG{n+nf}{fmi\PYGZus{}freeInstance}\PYG{p}{(}\PYG{k+kt}{void} \PYG{o}{*}\PYG{p}{)}\PYG{p}{;}

\PYG{c+cp}{\PYGZsh{}}\PYG{c+cp}{pragma pack(push, 8)}
\PYG{k}{typedef} \PYG{k}{struct} \PYG{n}{\PYGZus{}Stru\PYGZus{}Data\PYGZus{}Controller\PYGZus{}To\PYGZus{}Plant\PYGZus{}ex}
\PYG{p}{\PYGZob{}}
    \PYG{k+kt}{double} \PYG{n}{y}\PYG{p}{;}
    \PYG{k+kt}{double} \PYG{n}{z}\PYG{p}{;}
\PYG{p}{\PYGZcb{}}\PYG{n}{Stru\PYGZus{}Data\PYGZus{}Controller\PYGZus{}To\PYGZus{}Plant\PYGZus{}ex}\PYG{p}{;}

\PYG{k}{typedef} \PYG{k}{struct} \PYG{n}{\PYGZus{}Stru\PYGZus{}Data\PYGZus{}Controller\PYGZus{}To\PYGZus{}Plant\PYGZus{}ex1}\PYG{p}{\PYGZob{}}
    \PYG{k+kt}{double} \PYG{n}{y}\PYG{p}{;}
    \PYG{k+kt}{double} \PYG{n}{z}\PYG{p}{;}
\PYG{p}{\PYGZcb{}}\PYG{n}{Stru\PYGZus{}Data\PYGZus{}Controller\PYGZus{}To\PYGZus{}Plant\PYGZus{}ex1}\PYG{p}{;}
\PYG{k}{typedef} \PYG{k}{struct} \PYG{n}{\PYGZus{}Stru\PYGZus{}Data\PYGZus{}Controller\PYGZus{}To\PYGZus{}Plant\PYGZus{}ex2} \PYG{p}{\PYGZob{}}
    \PYG{k+kt}{double} \PYG{n}{y}\PYG{p}{;}
    \PYG{k+kt}{double} \PYG{n}{z}\PYG{p}{;}
\PYG{p}{\PYGZcb{}}\PYG{n}{Stru\PYGZus{}Data\PYGZus{}Controller\PYGZus{}To\PYGZus{}Plant\PYGZus{}ex2}\PYG{p}{;}
\PYG{k}{typedef} \PYG{k}{struct} \PYG{n}{\PYGZus{}Stru\PYGZus{}Data\PYGZus{}Controller\PYGZus{}To\PYGZus{}Plant\PYGZus{}ex3}   \PYG{p}{\PYGZob{}}
    \PYG{k+kt}{double} \PYG{n}{y}\PYG{p}{;}
    \PYG{k+kt}{double} \PYG{n}{z}\PYG{p}{;}
\PYG{p}{\PYGZcb{}}\PYG{n}{Stru\PYGZus{}Data\PYGZus{}Controller\PYGZus{}To\PYGZus{}Plant\PYGZus{}ex3}\PYG{p}{;}

\PYG{k}{typedef} \PYG{k}{struct} \PYG{n}{\PYGZus{}Stru\PYGZus{}Data\PYGZus{}Controller\PYGZus{}To\PYGZus{}Plant}\PYG{p}{\PYGZob{}}
    \PYG{k+kt}{double} \PYG{n}{F}\PYG{p}{;}
    \PYG{k+kt}{double} \PYG{n}{x\PYGZus{}0}\PYG{p}{;}
    \PYG{k+kt}{double} \PYG{n}{v\PYGZus{}0}\PYG{p}{;}
\PYG{p}{\PYGZcb{}}\PYG{n}{Stru\PYGZus{}Data\PYGZus{}Controller\PYGZus{}To\PYGZus{}Plant}\PYG{p}{;}

\PYG{k}{typedef} \PYG{k}{struct} \PYG{n}{\PYGZus{}Stru\PYGZus{}Data\PYGZus{}Plant\PYGZus{}To\PYGZus{}Controller}
\PYG{p}{\PYGZob{}}
    \PYG{k+kt}{double} \PYG{n}{x}\PYG{p}{;}
    \PYG{k+kt}{double} \PYG{n}{v}\PYG{p}{;}
\PYG{p}{\PYGZcb{}}\PYG{n}{Stru\PYGZus{}Data\PYGZus{}Plant\PYGZus{}To\PYGZus{}Controller}\PYG{p}{;}

\PYG{c+cp}{\PYGZsh{}}\PYG{c+cp}{pragma pack(pop)}

\PYG{c+cp}{\PYGZsh{}}\PYG{c+cp}{endif }\PYG{c+c1}{// INTERFACE\PYGZus{}H\PYGZus{}\PYGZus{}}
\end{sphinxVerbatim}


\chapter{Milestone的图形用户界面操作方式}
\label{\detokenize{gui/index:milestone}}\label{\detokenize{gui/index::doc}}

\section{新建模型并生成模板}
\label{\detokenize{gui/_u521b_u5efa_u6a21_u578b:id1}}\label{\detokenize{gui/_u521b_u5efa_u6a21_u578b::doc}}

\section{打开已有的模型头文件}
\label{\detokenize{gui/_u7f16_u8f91_u6a21_u578b:id1}}\label{\detokenize{gui/_u7f16_u8f91_u6a21_u578b::doc}}

\section{模型模板的代码结构及资源接口}
\label{\detokenize{gui/_u4ee3_u7801_u6a21_u677f:id1}}\label{\detokenize{gui/_u4ee3_u7801_u6a21_u677f::doc}}

\chapter{Milestone的命令行操作方式}
\label{\detokenize{cli/index:milestone}}\label{\detokenize{cli/index::doc}}

\section{添加模型}
\label{\detokenize{cli/index:id1}}
复制model内部的模型目录结构(内部sources文件夹为必须),实现与模型文件夹同名称的.h及.cpp模型代码文件
若增加新的模型间接口,在model/interface.h中定义接口数据结构体
在顶层CMakeLists.txt中”foreach (MODEL\_NAME controller plant plant\_1) \# add model to this list”语句处,将新的模型添加在列表中
重新执行上述构建操作,系统将执行增量构建


\chapter{FMI标准模型接口及其实现}
\label{\detokenize{fmi/index:fmi}}\label{\detokenize{fmi/index::doc}}

\chapter{S-Function模型接口及其实现}
\label{\detokenize{sfcn/index:s-function}}\label{\detokenize{sfcn/index::doc}}
生成fmu的过程中,在模型的sources目录中也生成了支持Simulink导入的S-Function接口代码
若要生成模型的S-Function模块,在顶层SFcnLists.m中”model\_list = \{‘controller’, ‘plant’\}; \% add model to this list”语句处,将新的模型添加在列表中
启动MATLAB,将工作路径切换至工具包根目录,运行SFcnLists.m脚本,将执行S-Function的代码生成和模块构建
保存获得的Simulink模型模块,以及工作空间中的数据总线定义,分发模型时还需要附加*.mexw32/{\color{red}\bfseries{}*}.mexw64二进制文件,以及模型所需的数据文件


\chapter{工具和索引}
\label{\detokenize{index:id1}}\begin{itemize}
\item {} 
\DUrole{xref,std,std-ref}{genindex}

\item {} 
\DUrole{xref,std,std-ref}{modindex}

\item {} 
\DUrole{xref,std,std-ref}{search}

\end{itemize}



\renewcommand{\indexname}{索引}
\printindex
\end{document}